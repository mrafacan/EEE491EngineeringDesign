
\documentclass[12pt]{article}
\usepackage[english]{babel}
\usepackage{natbib}
\usepackage{url}
\usepackage[utf8x]{inputenc}
\usepackage{enumitem}
\usepackage{amsmath}
\usepackage{graphicx}
\graphicspath{{images/}}
\usepackage{parskip}
\usepackage{fancyhdr}
\usepackage{vmargin}
\usepackage[section]{placeins}
\usepackage{multirow}
\usepackage{multicol}
\usepackage{lipsum}
\usepackage{blindtext}
\usepackage{amssymb}

\usepackage{float}
\usepackage{nameref}
\setmarginsrb{3 cm}{0.5 cm}{3 cm}{2.5 cm}{1 cm}{1.5 cm}{1 cm}{1.5 cm}
\title{EEE 491 Engineering Design I}							
\date{\today}											

\makeatletter
\let\thetitle\@title
\let\theauthor\@author
\let\thedate\@date
\makeatother

\pagestyle{fancy}
\fancyhf{}
\rhead{\theauthor}
\lhead{\thetitle}
\cfoot{\thepage}

\begin{document}
	
	%%%%%%%%%%%%%%%%%%%%%%%%%%%%%%%%%%%%%%%%%%%%%%%%%%%%%%%%%%%%%%%%%%%%%%%%%%%%%%%%%%%%%%%%%
	
	\begin{titlepage}
		\centering
		\vspace*{0.5 cm}
		\includegraphics[scale = 0.75]{images/gazi.png}\\[1.0 cm]	
		\textsc{\LARGE Gazi University}\\[0.5 cm]
		\textsc{\LARGE Faculty of Engineering}\\[0.5 cm]
		\textsc{\LARGE Electrical Electronics Engineering}\\[1 cm]
		\textsc{\Large ElGamal Encryption System}\\[0.5 cm]			
		\rule{\linewidth}{0.5 mm} \\[0.4 cm]
		{ \huge \bfseries \thetitle}\\
		\rule{\linewidth}{0.5 mm} \\[1.5 cm]
		
		\begin{minipage}{0.4\textwidth}
			\begin{flushleft} \large
				\emph{Student Name(s):}\\
				\text{Bora Bostanoğlu}\\
				\text{Eren Öztürk}\\
				\text{Mert Afacan}\\
				
			\end{flushleft}
			
		\end{minipage}~
		\begin{minipage}{0.4\textwidth}
			\begin{flushright} \large
				\emph{Student Number:} \\				
				\text{191112012}\\
				\text{191112029}\\
				\text{191112001}\\
			% Your Student Number
			\end{flushright}
		\end{minipage}\\[2 cm]
			\begin{center}
			\textbf{Prof. Dr. Erkan Afacan}
		\end{center}
		{\large \thedate}\\[2 cm]
		
		\vfill
		
	\end{titlepage}
	%%%%%%%%%%%%%%%%%%%%%%%%%%%%%%%%%%%%%%%%%%%%%%%%%%%%%%%%%%%%%%%%%%%%%%%%%%%%%%%%%%%%%%%%%
	\tableofcontents
	\pagebreak
	
	%%%%%%%%%%%%%%%%%%%%%%%%%%%%%%%%%%%%%%%%%%%%%%%%%%%%%%%%%%%%%%%%%%%%%%%%%%%%%%%%%%%%%%%%%
	\pagenumbering{Roman} % İçindekilerin Roma rakamlarıyla başlaması için
	
	
	\renewcommand{\abstractname}{Summary}
	\begin{abstract}
		\addcontentsline{toc}{section}{Summary}				
		In response to the growing need for accessible secure communication, this project introduces an innovative system leveraging ElGamal encryption to democratize confidential messaging. The aim is to bridge the gap between robust security and affordability, enabling individuals from diverse backgrounds to engage in secure text-based interactions. The project has achieved significant milestones, having developed the encryption algorithm and identified specific hardware components, including STM32 microcontrollers, a Wi-Fi module for data transmission, and USB interfaces for connectivity. The system operates through two STM32 development boards—one dedicated to encryption and the other for decryption—offering a cost-effective and simplified approach. By utilizing these readily available and widely compatible hardware components, this initiative strives to make secure communication an everyday reality, fostering a safer digital environment where privacy is prioritized. Ultimately, the goal is to redefine secure messaging, ensuring that individuals can interact confidently while maintaining the confidentiality of their conversations
		
	\end{abstract}
	

\clearpage % Yeni bir sayfa başlatmak için
\pagenumbering{arabic}
	\section{Introduction}
In today's digital age, ensuring the confidentiality and security of our communication is paramount. However, achieving this security often comes with a hefty price tag, rendering it inaccessible to many. Recognizing the need for a cost-effective yet robust solution, we introduce a system designed to democratize secure messaging using the efficient ElGamal encryption method.

The primary goal of this project is to establish a secure communication framework accessible to everyone. By leveraging ElGamal encryption, renowned for its effectiveness and affordability, we aim to break barriers and provide a platform for confidential text-based interactions without imposing significant financial burdens.

The system operates through a simplified yet potent mechanism involving two development boards. One board is tasked with encrypting the data, ensuring its confidentiality, while the other board handles decryption, allowing secure retrieval of the original message. This approach not only ensures affordability but also simplicity in its execution, making secure communication an achievable reality for individuals across diverse backgrounds.

Through this cost-efficient strategy, our system endeavors to facilitate secure message exchanges without compromising on the strength of encryption. By employing readily available development boards, we strive to make secure communication an integral part of everyday interactions, fostering a safer and more inclusive digital environment.

In essence, this project aims to redefine secure communication by bridging the gap between affordability and robust encryption. By democratizing access to secure messaging, we envision a future where individuals can interact confidently, knowing their conversations remain private and protected.


\vskip 20cm
	\section{Needs Identification}

For text-based encryption, there is a need to create a system that enables secure messaging for everyone. Utilizing the cost-effective ElGamal encryption method, this system aims to provide accessible and secure communication for individuals seeking confidential text-based interaction without incurring significant expenses. The system operates through two development boards, where one board encrypts the data and sends it to the other board, responsible for decryption. This cost-efficient approach ensures that anyone can securely exchange messages, fostering a safer digital environment for communication.

	
	\subsection{Statement of Engineering Problem:}
	The engineering challenge lies in developing a system that efficiently enables text-based encryption for secure messaging without imposing high costs. The existing solutions might either lack accessibility or be cost-prohibitive for many users. The problem at hand involves creating an encryption solution that ensures confidentiality in text-based communication without being financially burdensome.
	
	\subsection{Goal and Objectives:}
	\subsubsection{Goal:}
	The primary objective is to create a cost-efficient system utilizing the ElGamal encryption method, enabling accessible and secure text-based communication for all users.
	
	\subsubsection{Objectives:}
	\begin{itemize}
		\item Develop a system that implements ElGamal encryption for text-based communication securely.
		\item Ensure the system is cost-effective, avoiding significant financial barriers for users.
		\item Design the system to operate between two development boards, enabling encryption and decryption seamlessly.
		\item Provide a means for individuals to securely exchange messages without compromising confidentiality.
		\item Foster a safer digital environment by making secure communication accessible to everyone.
	\end{itemize}
\newpage
	\section{Research Survey and Background information}

	\begin{itemize}
		\item \textbf{ElGamal Encryption}:
		\begin{itemize}
			\item ElGamal is a public-key (asymmetric) cryptosystem based on the mathematical properties of modular exponentiation and the difficulty of certain mathematical problems.
			\item It involves key generation, encryption, and decryption processes.
		\end{itemize}
		
		\item \textbf{How It's Currently Being Done}:
		\begin{itemize}
			\item \textbf{Key Generation}: Each card generates its own public and private key pair.
			\item \textbf{Encryption}: Card A encrypts a message using Card B's public key and a random integer number.
			\item \textbf{Decryption}: Card B decrypts the received message using its private key.
		\end{itemize}
		
		\item \textbf{Limitations of Current Designs or Technology}:
		\begin{itemize}
			\item \textbf{Computational Intensity}: ElGamal encryption can be computationally intensive, which might be a limitation for resource-constrained devices.
			\item \textbf{Key Size}: To ensure security, ElGamal often requires larger key sizes, potentially impacting communication efficiency.
			\item \textbf{Vulnerability to Quantum Attacks}: ElGamal, like many public-key systems, is vulnerable to attacks leveraging quantum computers.
		\end{itemize}
	\end{itemize}
		
		
		In an article published by Omar A. Imran in 2019\cite{IMRAN20201028} they focused on speech signal encryption/decryption.	The referenced article primarily focuses on employing El-Gamal encryption for speech signal encryption and decryption. 
		 
	 \textbf{The differences of our project from the mentioned model are as follows:}
	 \begin{itemize}
	 	\item \textbf{Dual-Board Encryption System}: Unlike some existing projects that might focus on single-board encryption, our project distinctly involves the development of a dual-board system specifically designed for ElGamal encryption.
	 \end{itemize}
	 
	 \textbf{The similarities of our project with the mentioned article are as follows:}
	 \begin{itemize}
	 	\item \textbf{Focus on Data Security}: Similar to several existing projects, our project shares a fundamental focus on ensuring data security. Encryption and decryption processes aim to safeguard transmitted data from unauthorized access.
	 	
	 	\item \textbf{Utilization of ElGamal Algorithm}: Both the referenced article and our project utilize the ElGamal algorithm for encryption purposes, acknowledging its strength in public key cryptography.
	 \end{itemize}
		

		
	
	
	\section{Requirements Specifications}
		
	\subsection{Engineering requirements}
	\begin{enumerate}
		\item[a.] Shall  communicate wireless.
		\item[b.] Shall encrypt/decrypt using at least 32 bit prime numbers.
		\item[c.] Shall has at least 1 hours operating time.
		\item[d.] Shall determine how the El-Gamal encryption algorithm will be implemented on the development board. This shall involve the mathematical operations of the algorithm, memory usage, and processor requirements. 
		\item[e.] Shall address security aspects such as the random number generator and key management necessary for El-Gamal encryption in a sensitive and secure manner.
		
	
				
	\end{enumerate}
	\subsection{Marketing Requirements}
	\begin{enumerate}
		\item[a.] Shall be cost-effective in its implementation.
		\item[b.] Shall have a user-friendly interface or provide adequate user support.
		\item[c.] Shall be usable in concealing data in every desired text.
		\item[d.] Shall be portable and moduler.
		\item[e.] Shall be dependable in ensuring user data security.				
	\end{enumerate}
	\subsection{Constraints}
		\subsubsection{Sustainability}
		ElGamal Encryption may involve complex mathematical operations, and the algorithms used for encryption and decryption can consume significant computational resources. It's important to consider the sustainability of the system in terms of energy efficiency, especially if it is deployed in resource-constrained environments.
		\subsubsection{Environmental}
		Our environmental goal in this encryption project is to mitigate collateral damage associated with vulnerability that can affect environment. While encryption is essential for data security, it can inadvertently lead to collateral risks such as vulnerabilities or access limitations. Our focus is mitigate any potential adverse effects that might arise as collateral damage.
		\subsubsection{Economical}
		The ElGamal encryption implementation requires the use of cost-effective components crucial for encryption and communication purposes. This economic analysis aims to outline the cost estimates associated with these components, emphasizing the utilization of budget-friendly options.
			
		The ElGamal encryption project's economic aspect estimates a total cost of 20\$ for the development boards. Emphasis on utilizing budget-friendly components is highlighted for an economical solution. Further detailing of any additional expenses is recommended for a comprehensive economic evaluation.
		\subsubsection{Manufacturability}
	Constraints related to manufacturability should address the ease of integrating ElGamal Encryption into existing systems and technologies. Ensuring compatibility and interoperability with different platforms and software solutions is crucial for seamless adoption and widespread use.
		\subsubsection{Social}
  		Our project is subject to various social constraints such as respect for privacy, user consent, ethical use and security measures. The technologies used in the project must comply with legal regulations, and the transparent processing and protection of user data must be prioritized. User awareness should be raised through educational materials and community engagement.
	\subsection{Certifications and/or Standards}
	In our pursuit of developing a robust ElGamal encryption solution, we are dedicated to integrating industry-established standards into our hardware and embedded software development practices. The incorporation of IEEE 1413 for hardware description languages and IEEE/ISO/IEC 14764 for software engineering processes serves as the cornerstone of our commitment. IEEE 1413 guides our hardware design methodologies, ensuring consistency and efficiency in utilizing hardware description languages for electronic systems. Simultaneously, IEEE/ISO/IEC 14764 shapes our software engineering processes, emphasizing meticulous software maintenance, management, and configuration control practices throughout the lifecycle. By adhering to these standards, we aim to fortify the reliability, compatibility, and sustainability of our ElGamal encryption project, assuring a high-quality, industry-aligned solution.
	\section{Concept Generation and Evaluation}
	
	\subsection{Level 0 Design:}
	\begin{figure}[H]
		\centering
		\label{Level 0 Design of System }
		\includegraphics[scale = 0.15]{images/level0design.jpg}\\[0.5 cm]	
		\caption{Level 0 Design of System } 		
	\end{figure}
	\begin{table}[H]
		\centering
		
		\label{Properties Of Level 0 Design }
		\begin{tabular}{|c|c|}
			\hline
			Module & ElGamal Encryption/Decryption System \\ \hline
			\multirow{2}{*}{Inputs} & Power 3.3V DC \\ \cline{2-2}
			& User Input (Message) \\ \hline
			Output & Decrypted Message \\ \hline
			Functionality & Encrypting given input and  decrypting encrypted input \\ \hline
		
		\end{tabular}
		\caption{Properties of Level 0 Design }
	\end{table}
	
	\subsection{Level 1 Design:}

	\begin{figure}[H]
		\centering
		\label{Level 1 Design of System }
		\includegraphics[scale = 0.25]{images/level1design.jpg}\\[0.5 cm]	
		\caption{Level 1 Design of System } 		
	\end{figure}
	

\begin{table}[h]

	\centering
	
	\label{Properties of Encryption Card}
	\begin{tabular}{|c|c|}
		\hline
		Module & Encryption Card \\ \hline
		\multirow{3}{*}{Inputs} & Power 3.3V DC \\
		\cline{2-2}
		& User Input (Message) \\
		\cline{2-2}
		& Public Key \\ \hline
		Output & Encrypted Message \\ \hline
		Functionality & Encrypting given input.\\ \hline		
	\end{tabular}
	\caption{Properties of Encryption Card}

\end{table}

\begin{table}[h]

	\centering	
	\label{Properties of Decryption Card }
	\begin{tabular}{|c|c|}
	\hline
	Module & Decryption Card\\ \hline
		\multirow{2}{*}{Inputs} & Power 3.3V DC \\ \cline{2-2}
	& Encrypted Message \\ \hline
		\multirow{2}{*}{Outputs} & Public Key \\ \cline{2-2}
	& Decrypted Message \\ \hline
	Functionality & Decrypting given input.\\ \hline	
		
	\end{tabular}
	\caption{Properties of Decryption Card}

\end{table}
	\begin{table}[H]

		\centering		
		\label{Properties of Access Point }
		\begin{tabular}{|c|c|}
			\hline
			Module & Access Point \\ \hline
			\multirow{3}{*}{Inputs} & Power 3.3V DC \\
			
			\cline{2-2}
				& Encrypted Message \\
		
			\cline{2-2}
				& Public Key \\
			\cline{2-2} \hline
	
				\multirow{2}{*}{Outputs} 	& Public Key \\
			\cline{2-2}
			& Encrypted Message \\
			
			\cline{2-2} \hline
			
			Functionality & Access Point.\\ \hline	
	
		\end{tabular}
		\caption{Properties of Access Point}

	\end{table}
	
	\begin{table}[H]
	\centering
	\begin{tabular}{|c|c|c|c|c|}
		\hline
		UI & Display & Connection & Power & Dev Board \\
		\hline
		\textbf{UI App}$\checkmark$
		 &\textbf{Monitor}$\checkmark$
		   & \textbf{Wifi}$\checkmark$
		   & Solar Cell & Arduino\\
		\hline
		\textbf{Keyboard}$\checkmark$
		 & LCD & Bluetooth & Li-on Battery & Rasberry Pi \\
		\hline
		Voice & Head-up Display & Ethernet & Fuel Cell &\textbf{Stm32}$\checkmark$
		 \\
		\hline
		Video & Oled & USB & \textbf{Lipo Battery}$\checkmark$
		 & Jetson Nano \\
		\hline
	\end{tabular}
	\caption{Consept Table}
\end{table}


\vskip 10cm
\subsection{Demonstration of how behavioral models:}
	
	\subsubsection{Activity Diagram}
 	The general UML diagram of our software is shown in the figure and it shows that. First, a public key is generated on the decryption card for encryption and this key is sent to the encryption card. Then, on the encryption card, the user's message is converted into a numeric value and encrypted with the public key of the decryption card and the encrypted message is sent to the decryption card. The encrypted message is decrypted with the decryption card's own private key and the original message is obtained on the decryption card.
 	
 	\begin{figure}[H]
 		\centering
 		\label{Uml Diagram Of The System}
 		\includegraphics[scale = 0.35]{images/activitydiagram.jpg}\\[0.5 cm]	
 		\caption{Activity Diagram Of The System} 		
 	\end{figure}
 	\newpage
	\section{Conclusions}
	
	Firstly, it is crucial to emphasize the purpose and requirements of the project. We plan to use two development boards to implement ElGamal encryption. One of these boards will perform the encryption process, while the other will decrypt the cipher. The user's input text will be encrypted for secure transmission. The primary objectives of our project include being cost-effective, appealing to a wide audience, and ensuring accessibility.
	
	Our proposed solution will employ the secure encryption algorithm ElGamal to ensure the security of communication. This differentiates our solution from other available alternatives because ElGamal is generally acknowledged as a reliable and effective encryption method. Additionally, we offer a hardware-based solution using two different development boards, which can be considered an innovative approach to the project.
	
	The analyses conducted indicate that the project provides a cost-effective solution and is accessible to a broad user base. ElGamal encryption stands out with its reliability and effectiveness. The hardware-based solution may offer faster and more reliable performance.
	
	In conclusion, our designed solution has significant advantages, such as reliability, cost-effectiveness, and appeal to a broad user base. The use of the ElGamal encryption algorithm is seen as an effective method for ensuring communication security. Our hardware-based approach also provides a specific advantage to the project. Overall, this project emerges as a suitable solution for general use.
	\newpage
	\section{References}


\bibliographystyle{plain}
\bibliography{S1877050920308681} % Burada "referanslar" dosyasının adı

	\newpage
	\section{Appendix}
	\begin{figure}[H]
	\centering
	\label{Meeting1}
	\includegraphics[scale = 0.3]{images/meet3.png}\\[0.5 cm]			
\end{figure}
	\begin{figure}[H]
	\centering
	\label{Meeting2}
	\includegraphics[scale = 0.3]{images/meet2.png}\\[0.5 cm]			
\end{figure}
	\begin{figure}[H]
	\centering
	\label{Meeting3}
	\includegraphics[scale = 0.3]{images/meet1.png}\\[0.5 cm]			
\end{figure}
	
	

\end{document}
