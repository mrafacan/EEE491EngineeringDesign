
\documentclass[12pt]{article}
\usepackage[english]{babel}
\usepackage{natbib}
\usepackage{url}
\usepackage[utf8x]{inputenc}
\usepackage{enumitem}
\usepackage{amsmath}
\usepackage{graphicx}
\usepackage{colortbl}
\usepackage[table]{xcolor}
\graphicspath{{images/}}
\usepackage{parskip}
\usepackage{fancyhdr}
\usepackage{vmargin}
\usepackage[section]{placeins}
\usepackage{multirow}
\usepackage{multicol}
\usepackage{lipsum}
\usepackage{blindtext}
\usepackage{amssymb}
\usepackage[nottoc,numbib]{tocbibind}
\usepackage{float}
\usepackage{nameref}
\usepackage{longtable}
\setmarginsrb{3 cm}{0.5 cm}{3 cm}{2.5 cm}{1 cm}{1.5 cm}{1 cm}{1.5 cm}
\title{EEE 491 Engineering Design I}							
\date{\today}											

\makeatletter
\let\thetitle\@title
\let\theauthor\@author
\let\thedate\@date
\makeatother

\pagestyle{fancy}
\fancyhf{}
\rhead{\theauthor}
\lhead{\thetitle}
\cfoot{\thepage}

\begin{document}
	
	%%%%%%%%%%%%%%%%%%%%%%%%%%%%%%%%%%%%%%%%%%%%%%%%%%%%%%%%%%%%%%%%%%%%%%%%%%%%%%%%%%%%%%%%%
	
	\begin{titlepage}
		\centering
		\vspace*{0.5 cm}
		\includegraphics[scale = 2.85]{images/gazi.png}\\[1.0 cm]	
		\textsc{\LARGE Gazi University}\\[0.5 cm]
		\textsc{\LARGE Faculty of Engineering}\\[0.5 cm]
		\textsc{\LARGE Electrical Electronics Engineering}\\[1 cm]
		\textsc{\Large ElGamal Encryption System}\\[0.5 cm]			
		\rule{\linewidth}{0.5 mm} \\[0.4 cm]
		{ \huge \bfseries \thetitle}\\
		\rule{\linewidth}{0.5 mm} \\[1.5 cm]
		
		\begin{minipage}{0.4\textwidth}
			\begin{flushleft} \large
				\emph{Student Name(s):}\\
				\text{Mert Afacan}\\
				\text{Bora Bostanoğlu}\\
				\text{Eren Öztürk}\\
			
				
			\end{flushleft}
			
		\end{minipage}~
		\begin{minipage}{0.4\textwidth}
			\begin{flushright} \large
				\emph{Student Number:} \\	
				\text{191112001}\\			
				\text{191112012}\\
				\text{191112029}\\
				
			% Your Student Number
			\end{flushright}
		\end{minipage}\\[2 cm]
			\begin{center}
			\textbf{Prof. Dr. Erkan Afacan}
		\end{center}
		{\large \thedate}\\[2 cm]
		
		\vfill
		
	\end{titlepage}
	%%%%%%%%%%%%%%%%%%%%%%%%%%%%%%%%%%%%%%%%%%%%%%%%%%%%%%%%%%%%%%%%%%%%%%%%%%%%%%%%%%%%%%%%%
	\tableofcontents
	\thispagestyle{empty}
	\pagebreak
	
	%%%%%%%%%%%%%%%%%%%%%%%%%%%%%%%%%%%%%%%%%%%%%%%%%%%%%%%%%%%%%%%%%%%%%%%%%%%%%%%%%%%%%%%%%
	\pagenumbering{Roman} % İçindekilerin Roma rakamlarıyla başlaması için
	
	
	\renewcommand{\abstractname}{Summary}
	\begin{abstract}
		\addcontentsline{toc}{section}{Summary}				
		With the goal of bridging the gap between robust security and affordability, this project introduces an innovative system that uses ElGamal encryption to make available confidential messaging in response to the growing demand for accessible secure communication. This system allows individuals from diverse backgrounds to participate in secure text-based interactions. After developing the encryption algorithm and identifying particular hardware components—such as STM32 microcontrollers, a Wi-Fi module for data transmission, and USB interfaces for connectivity—the project has reached important milestones. The system runs on two STM32 development boards, one for encryption and the other for decryption, providing an efficient and streamlined method.The ultimate goal is to redefine secure messaging, guaranteeing that people can interact with confidence while protecting the confidentiality of their conversations. This initiative looks to make secure communication a daily reality by utilizing these widely compatible and easily accessible hardware components.
		
	\end{abstract}
	

\clearpage % Yeni bir sayfa başlatmak için
\pagenumbering{arabic}
	\section{Introduction}
In our era, ensuring the confidentiality and security of our communication is important. However, achieving this security often comes with a unaffordable price tag, makes it inaccessible to many. Recognizing the need for a cost-effective yet robust solution, we introduce a system designed to make secure messaging accessible to everyone using the efficient ElGamal encryption method.

This project's main objective is to create a safe communication environment that is open to all. We do this by utilizing ElGamal encryption, which is widely known for its efficiency and low cost. We hope to remove obstacles and offer a platform for private text messages without putting a heavy financial strain on users.

Two development boards function as the system's basic yet efficient mechanism: one board encrypts the data, guaranteeing its privacy, and the other board decrypts it, enabling safe delivery of the initial message. This method guarantees both affordability and ease of use, enabling secure communication to be a reality for people from a variety of backgrounds.

By utilizing these affordable development boards, we hope to increase the security of messages sent and received while maintaining the integrity of the encryption. In this way, we hope to integrate secure communication into daily interactions and promote a more inclusive and safe online community.

At its core, our project seeks to reimagine secure communication by filling the gap that exists between effective encryption and affordability. We foresee a future in which people can engage with confidence, knowing that their messages are private and secure, because secure messaging will become more widely accessible.


\vskip 20cm
	\section{Needs Identification}

For text-based encryption, there is a need to create a system that enables secure messaging for everyone. Utilizing the cost-effective ElGamal encryption method, this system aims to provide accessible and secure communication for individuals seeking confidential text-based interaction without incurring significant expenses. The system operates through two development boards, where one board encrypts the data and sends it to the other board, responsible for decryption. This cost-efficient approach ensures that anyone can securely exchange messages, fostering a safer digital environment for communication.

	
	\subsection{Statement of Engineering Problem:}
	The engineering challenge lies in developing a system that efficiently enables text-based encryption for secure messaging without imposing high costs. The existing solutions might either lack accessibility or be cost-prohibitive for many users. The problem at hand involves creating an encryption solution that ensures confidentiality in text-based communication without being financially burdensome.
	
	\subsection{Goal and Objectives:}
	\subsubsection{Goal:}
	The primary objective is to create a cost-efficient system utilizing the ElGamal encryption method, enabling accessible and secure text-based communication for all users.
	
	\subsubsection{Objectives:}
	\begin{itemize}
		\item Develop a system that implements ElGamal encryption for text-based communication securely.
		\item Ensure the system is cost-effective, avoiding significant financial barriers for users.
		\item Design the system to operate between two development boards, enabling encryption and decryption seamlessly.
		\item Provide a means for individuals to securely exchange messages without compromising confidentiality.
		\item Foster a safer digital environment by making secure communication accessible to everyone.
	\end{itemize}
\newpage
	\section{Research Survey and Background information}

	\begin{itemize}
		\item \textbf{ElGamal Encryption}:
		\begin{itemize}
			\item ElGamal is a public-key (asymmetric) cryptosystem based on the mathematical properties of modular exponentiation and the difficulty of certain mathematical problems\cite{book}.
			\item It involves key generation, encryption, and decryption processes.
		\end{itemize}
		
		\item \textbf{How It's Currently Being Done}:
		\begin{itemize}
			\item \textbf{Key Generation}: Each card generates its own public and private key pair.
			\item \textbf{Encryption}: Card A encrypts a message using Card B's public key and a random integer number.
			\item \textbf{Decryption}: Card B decrypts the received message using its private key.
		\end{itemize}
		
		\item \textbf{Limitations of Current Designs or Technology}:
		\begin{itemize}
			\item \textbf{Computational Intensity}: ElGamal encryption can be computationally intensive, which might be a limitation for resource-constrained devices.
			\item \textbf{Key Size}: To ensure security, ElGamal often requires larger key sizes, potentially impacting communication efficiency.
			\item \textbf{Vulnerability to Quantum Attacks}: ElGamal, like many public-key systems, is vulnerable to attacks leveraging quantum computers.
		\end{itemize}
	\end{itemize}
		
		
		In an article published by Omar A. Imran in 2019\cite{IMRAN20201028} they focused on speech signal encryption/decryption.	The referenced article primarily focuses on employing El-Gamal encryption for speech signal encryption and decryption. 
		 
	 \textbf{The differences of our project from the mentioned model are as follows:}
	 \begin{itemize}
	 	\item \textbf{Dual-Board Encryption System}: Unlike some existing projects that might focus on single-board encryption, our project distinctly involves the development of a dual-board system specifically designed for ElGamal encryption.
	 \end{itemize}
	 
	 \textbf{The similarities of our project with the mentioned article are as follows:}
	 \begin{itemize}
	 	\item \textbf{Focus on Data Security}: Similar to several existing projects, our project shares a fundamental focus on ensuring data security. Encryption and decryption processes aim to safeguard transmitted data from unauthorized access.
	 	
	 	\item \textbf{Utilization of ElGamal Algorithm}: Both the referenced article and our project utilize the ElGamal algorithm for encryption purposes, acknowledging its strength in public key cryptography.
	 \end{itemize}
		

		
	
	
	\section{Requirements Specifications}
		
	\subsection{Engineering requirements}
	\begin{enumerate}
		\item[a.] Shall communicate eachother wireless.
		\item[b.] Shall communicate users with USB.
		\item[c.] Shall encrypt/decrypt using at least 32 bit prime numbers.
		\item[d.] Shall has at least 1 hours operating time.
		\item[e.] Shall determine how the El-Gamal encryption algorithm will be implemented on the development board. This shall involve the mathematical operations of the algorithm, memory usage, and processor requirements. 
		\item[f.] Shall address security aspects such as the random number generator and key management necessary for El-Gamal encryption in a sensitive and secure manner.
		
	
				
	\end{enumerate}
	\subsection{Marketing Requirements}
	\begin{enumerate}
		\item[a.] Shall  cost-effective in its implementation.
		\item[b.] Shall have a user-friendly interface or provide adequate user support.
		\item[c.] Shall  usable in concealing data in every desired text.
		\item[d.] Shall  portable and moduler.
		\item[e.] Shall  dependable in ensuring user data security.				
	\end{enumerate}
	\subsection{Constraints}
		\subsubsection{Sustainability}
		ElGamal Encryption may involve complex mathematical operations, and the algorithms used for encryption and decryption can consume significant computational resources. It's important to consider the sustainability of the system in terms of energy efficiency, especially if it is deployed in resource-constrained environments.
		\subsubsection{Environmental}
		Our environmental goal in this encryption project is to mitigate collateral damage associated with vulnerability that can affect environment. While encryption is essential for data security, it can inadvertently lead to collateral risks such as vulnerabilities or access limitations. Our focus is mitigate any potential adverse effects that might arise as collateral damage.
		\subsubsection{Economical}
		The ElGamal encryption implementation requires the use of cost-effective components crucial for encryption and communication purposes. This economic analysis aims to outline the cost estimates associated with these components, emphasizing the utilization of budget-friendly options.
			
		The ElGamal encryption project's economic aspect estimates a total cost of 20\$ for the development boards. Emphasis on utilizing budget-friendly components is highlighted for an economical solution. Further detailing of any additional expenses is recommended for a comprehensive economic evaluation.
		\subsubsection{Manufacturability}
	Constraints related to manufacturability should address the ease of integrating ElGamal Encryption into existing systems and technologies. Ensuring compatibility and interoperability with different platforms and software solutions is crucial for seamless adoption and widespread use.
		\subsubsection{Social}
  		Our project is subject to various social constraints such as respect for privacy, user consent, ethical use and security measures. The technologies used in the project must comply with legal regulations, and the transparent processing and protection of user data must be prioritized. User awareness should be raised through educational materials and community engagement.
	\subsection{Certifications and/or Standards}
	In our pursuit of developing a robust ElGamal encryption solution, we are dedicated to integrating industry-established standards into our hardware and embedded software development practices. The incorporation of IEEE 1413 for hardware description languages, ISO/IEC 27001 international standard to manage information security and IEEE/ISO/IEC 14764 for software engineering processes serves as the cornerstone of our commitment. IEEE 1413 guides our hardware design methodologies, ensuring consistency and efficiency in utilizing hardware description languages for electronic systems. Simultaneously, IEEE/ISO/IEC 14764 shapes our software engineering processes, emphasizing meticulous software maintenance, management and configuration control practices throughout the lifecycle. By adhering to these standards, we aim to fortify the reliability, compatibility and sustainability of our ElGamal encryption project, assuring a high-quality, industry-aligned solution.
	\section{Concept Generation and Evaluation}
	
	\subsection{Level 0 Design:}
	\begin{figure}[H]
		\centering
		\label{Level 0 Design of System }
		\includegraphics[scale = 0.15]{images/level0design.jpg}\\[0.5 cm]	
		\caption{Level 0 Design of System } 		
	\end{figure}
	\begin{table}[H]
		\centering
		
		\label{Properties Of Level 0 Design }
		\begin{tabular}{|c|c|}
			\hline
			Module & ElGamal Encryption/Decryption System \\ \hline
			\multirow{2}{*}{Inputs} & Power 3.3V DC \\ \cline{2-2}
			& User Input (Message) \\ \hline
			Output & Decrypted Message \\ \hline
			Functionality & Encrypting given input and  decrypting encrypted input \\ \hline
		
		\end{tabular}
		\caption{Properties of Level 0 Design }
	\end{table}
	
	\subsection{Level 1 Design:}

	\begin{figure}[H]
		\centering
		\label{Level 1 Design of System }
		\includegraphics[scale = 0.25]{images/level1designnew.jpg}\\[0.5 cm]	
		\caption{Level 1 Design of System } 		
	\end{figure}
	

\begin{table}[h]

	\centering
	
	\label{Properties of Encryption Card}
	\begin{tabular}{|c|c|}
		\hline
		Module & Encryption Card \\ \hline
		\multirow{3}{*}{Inputs} & Power 3.3V DC \\
		\cline{2-2}
		& User Input (Message) \\
		\cline{2-2}
		& Public Key \\ \hline
		Output & Encrypted Message \\ \hline
		Functionality & Encrypting given input.\\ \hline		
	\end{tabular}
	\caption{Properties of Encryption Card}

\end{table}

\begin{table}[h]

	\centering	
	\label{Properties of Decryption Card }
	\begin{tabular}{|c|c|}
	\hline
	Module & Decryption Card\\ \hline
		\multirow{2}{*}{Inputs} & Power 3.3V DC \\ \cline{2-2}
	& Encrypted Message \\ \hline
		\multirow{2}{*}{Outputs} & Public Key \\ \cline{2-2}
	& Decrypted Message \\ \hline
	Functionality & Decrypting given input.\\ \hline	
		
	\end{tabular}
	\caption{Properties of Decryption Card}

\end{table}
	\begin{table}[H]

		\centering		
		\label{Properties of Access Point }
		\begin{tabular}{|c|c|}
			\hline
			Module & Access Point \\ \hline
			\multirow{3}{*}{Inputs} & Power 3.3V DC \\
			
			\cline{2-2}
				& Encrypted Message \\
		
			\cline{2-2}
				& Public Key \\
			\cline{2-2} \hline
	
				\multirow{2}{*}{Outputs} 	& Public Key \\
			\cline{2-2}
			& Encrypted Message \\
			
			\cline{2-2} \hline
			
			Functionality & Access Point.\\ \hline	
	
		\end{tabular}
		\caption{Properties of Access Point}

	\end{table}
	
	\begin{table}[H]
	\centering
	\begin{tabular}{|c|c|c|c|c|}
		\hline
		UI & Display & Connection & Power & Dev Board \\
		\hline
		\textbf{UI App}$\checkmark$
		 &\textbf{Monitor}$\checkmark$
		   & \textbf{Wifi}$\checkmark$
		   & Solar Cell & Arduino\\
		\hline
		\textbf{Keyboard}$\checkmark$
		 & LCD & Bluetooth & Li-on Battery & Rasberry Pi \\
		\hline
		Voice & Head-up Display & Ethernet & Fuel Cell &\textbf{Stm32}$\checkmark$
		 \\
		\hline
		Video & Oled &\textbf{USB-TTL}$\checkmark$  & \textbf{Lipo Battery}$\checkmark$
		 & Jetson Nano \\
		\hline
	\end{tabular}
	\caption{Consept Table}
\end{table}


\vskip 10cm


\subsection{Level2 Design}
	\begin{figure}[H]
	\centering
	\label{Level2 Design Of The System}
	\includegraphics[scale = 0.25]{images/level2design.jpg}\\[0.5 cm]	
	\caption{Level2 Design Of The System} 		
\end{figure}
	\begin{figure}[H]
	\centering
	\label{Level21 Design Of The System}
	\includegraphics[scale = 0.25]{images/level2design1.jpg}\\[0.5 cm]	
	\caption{Level2 Design Of The System} 		
\end{figure}
	\begin{figure}[H]
	\centering
	\label{Level22 Design Of The System}
	\includegraphics[scale = 0.25]{images/level2design12.jpg}\\[0.5 cm]	
	\caption{Level2 Design Of The System} 		
\end{figure}
\newpage
\subsection{Test Plans}
\subsubsection*{Unit Test}
\begin{itemize}
	\item \textbf{USB TTL Module Test:}
	\begin{itemize}
		\item \textbf{Purpose:} To verify that the USB TTL module is capable of transmitting and receiving data correctly.
		\item \textbf{Procedure:} Send a predefined data sequence to the USB TTL module and confirm that the same sequence is received without errors.
		\item \textbf{Expected Result:} Data transmitted matches data received.
	\end{itemize}
	
	\item \textbf{Client Application Test:}
	\begin{itemize}
		\item \textbf{Purpose:} To ensure the client application on both the encryption and decryption systems processes input and output correctly.
		\item \textbf{Procedure:} Provide input to the client application and verify the response matches expected behavior for both encryption and decryption.
		\item \textbf{Expected Result:} The client application should correctly encrypt or decrypt the input data and display the expected results.
	\end{itemize}
	
	\item \textbf{Development Board Tests:}
	\begin{itemize}
		\item \textbf{Purpose:} To test the functionality of the development board in handling encryption and decryption tasks.
		\item \textbf{Procedure:} Load test firmware onto the development board that performs encryption and decryption tasks, using known keys and data.
		\item \textbf{Expected Result:} The development board successfully encrypts and decrypts data, matching expected output.
	\end{itemize}
\end{itemize}
\subsubsection*{Module Test}
\begin{itemize}
	\item \textbf{Client to Development Board Communication Test:}
	\begin{itemize}
		\item \textbf{Purpose:} To validate the communication link between the client application and the development board.
		\item \textbf{Procedure:} Execute a series of commands from the client to the development board and verify actions are executed as intended.
		\item \textbf{Expected Result:} Seamless communication and correct action execution.
	\end{itemize}
	
	\item \textbf{USB TTL to Development Board Interface Test:}
	\begin{itemize}
		\item \textbf{Purpose:} To check the interface between the USB TTL module and the development board for data integrity.
		\item \textbf{Procedure:} Send varying lengths of data packets and validate integrity upon receipt.
		\item \textbf{Expected Result:} No data corruption or loss in transmission.
	\end{itemize}
\end{itemize}

\subsubsection*{Functional Test}

	\begin{itemize}
		\item \textbf{Purpose:} To test the complete functionality of the encryption system.
		\item \textbf{Procedure:} Simulate user input and validate encrypted output against known encryption algorithms.
		\item \textbf{Expected Result:} Encrypted output matches the expected encrypted data.
	\end{itemize}
	
	\begin{itemize}
		\item \textbf{Purpose:} To confirm that the decryption system accurately deciphers encrypted messages.
		\item \textbf{Procedure:} Provide known encrypted data to the system and verify if the decrypted output matches the original plaintext.
		\item \textbf{Expected Result:} The system successfully decrypts the input to its original form.
	\end{itemize}

\subsubsection*{Integration Tests}

	\begin{itemize}
		\item \textbf{Purpose:} To validate the integrated performance of the encryption and decryption systems using the access point.
		\item \textbf{Procedure:} Send a message from User 1 through the encryption system to the access point, then to the decryption system, and finally to User 2. Verify the integrity of the received message at User 2's end.
		\item \textbf{Expected Result:} User 2 receives the message exactly as sent by User 1, demonstrating successful end-to-end encryption and decryption.
	\end{itemize}
	
	\begin{itemize}
		\item \textbf{Purpose:} To test the communication reliability and data integrity of the access point when interfacing with both the encryption and decryption systems.
		\item \textbf{Procedure:} Transmit encrypted data through the access point to the decryption system and verify consistency and integrity.
		\item \textbf{Expected Result:} The access point correctly relays data without loss or corruption.
	\end{itemize}

\newpage
\subsection{Demonstration of how behavioral models:}
	
	\subsubsection{Activity Diagram}
	
 The general activity diagram of our project is shown in the Figure 3 and it shows that. First, a public key is generated on the decryption card for encryption and this key is sent to the encryption card. Meanwhile user can enter a message to the user interface and send it to the encryption card. Then, on the encryption card, the user's message is converted into a numeric value and encrypted with the public key of the receiver card and the encrypted message is sent to the decryption card. on the decryption card, the encrypted message is decrypted with the decryption card's own private key and the original message is obtained. Finally decryption card sends message to the user interface and user can display the message.
 
 	\begin{figure}[H]
 		\centering
 		\label{Uml Diagram Of The System}
 		\includegraphics[scale = 0.3]{images/activitydiagram.jpg}\\[0.5 cm]	
 		\caption{Activity Diagram Of The System} 		
 	\end{figure}
 \newpage
 	\begin{figure}[H]
 	\centering
 	\label{Enc Diagram}
 	\includegraphics[scale = 0.35]{images/activitydiagramenc.jpg}\\[0.5 cm]	
 	\caption{Activity Diagram Of The Encryption System } 		
 \end{figure}
\begin{itemize}
	\item \textbf{bool SystemCheck():} This function verifies whether there is an issue within the system.
	\item \textbf{bool CheckPublicKey():} This function awaits public key
	\item \textbf{void Send(param1, param2):} 
	This function correctly delivers the message to the intended destination based on the two provided parameters.
	\item \textbf{bool CheckUserInput():}  This function awaits user input.
	\item \textbf{void CalcNumEq(userInput):} This function calculates the numerical equivalent of user input.
	\item \textbf{void Encrypt(numEq):} This function encrypts the numerical equivalent.
	\item \textbf {bool IsMessageSent():} This function checks if the message has been sent.
\end{itemize}
 \newpage
\begin{figure}[H]
	\centering
	\label{dec Diagram}
	\includegraphics[scale = 0.35]{images/activitydiagramdec.jpg}\\[0.5 cm]	
	\caption{Activity Diagram Of The Decryption System } 		
\end{figure}
\begin{itemize}
	\item \textbf{bool SystemCheck():}  This function verifies whether there is an issue within the system.
	\item \textbf{bool GenerateKey():} This function generates public and private keys.
	\item \textbf{void Send(publicKey, type):} 	This function correctly delivers the message to the intended destination based on the two provided parameters.
	\item \textbf{void Decrypt(encMes):} This function decrypts the incoming encrypted numerical equivalent.
	\item \textbf{void NumToText(decMes):} This function converts the decrypted numerical equivalent into text.
	\item \textbf{void Send(textMes, type):} This function correctly delivers the message to the intended destination based on the two provided parameters.
	\item \textbf {bool IsMessageSent()} This function checks if the message has been sent.
\end{itemize}
 \newpage
 \begin{figure}[H]
 	\centering
 	\label{}
 	\includegraphics[scale = 0.35]{images/datadiagram.png}\\[0.5 cm]	
 	\caption{Udp Diagram Of The System} 		
 \end{figure}

 \begin{table}[H]	
 	\centering
 	\begin{tabular}{|c|c|}
 		\hline
 		Data Name & Start Byte  \\
 		\hline
 		Public Key &  0xAA \\ \hline
 		Encrypted Message &  0xBB \\ \hline
 		Decrypted Message &  0xCC  \\ \hline
 		User Input & 0xDD  \\ \hline
 		Communication Error &  0xEE \\ \hline
 	\end{tabular}
	\caption{Start Byte's} 
\end{table}
 	\newpage
 	\section{Work Plan}
	\begin{table}[H]

 		\renewcommand{\arraystretch}{0.2}
 	\centering
 	\begin{longtable}{|p{1cm}|p{2.75cm}|p{1.6cm}|p{2.5cm}|p{1.25cm}|p{1.5cm}|p{2cm}|}	\hline
 		Task Id & Task Description & Task Owner & Task Responsibility & Task Status & Estimated Cost & Timeline \\
 		\hline
 		\endfirsthead
		
 		1 & Gathering Information &  &  &  & & \\  \hline
 		1.1 & Literature Research & Everyone & To have knowledge in the field of encryption. & Done &  & \\  \hline
 		2 & Main Components &  &  &  &  & \\  \hline
 		2.1 & Development Board & Eren & Selection of the appropriate card and  examining the datasheet.& Done & 331.85 * 2 =662.7  & 2 days \\  \hline
 		2.2 & Communication Module & Mert & Selection of the appropriate card and  examining the datasheet.  & Done  & 60.14 * 3 =180.42  & 2 days  \\  \hline
 		3 & Design States &  &  &  &  & \\  \hline
 		3.1 & Algorithm Design & Everyone & Fundamental Design of the Project  & Done  & 0  & 4 week  \\  \hline
 		3.2 & Software Design for Algorithm & Bora  & Reflecting the Algorithm to the code lines  & Not Started  & 0 & 4 week\\  \hline
 		3.3& Software Design for UI & Mert  & Designing UI for using the system. & Not Started  & 0  & 4 week \\  \hline
 		4 & Implementation &   &  &   &   & 4 week \\  \hline
 		4.1 & Purchase of other Components & Bora  & USB – TTL cables etc.  & Not Started  & 0 & 2 days  \\  \hline
 		4.2 & Test Debugging& Eren  & Validation  & Not Started  & 0  & 4 weeks  \\
 		\hline
 	
 	\end{longtable}
 		\caption{Work Plan Table}
\end{table}
 

 	\newpage
 	\section{Risk Analysis}
 	 	\centering
 	\begin{table}[H]
 	\centering
	\renewcommand{\arraystretch}{0.2}
	
 		\begin{longtable}{|p{3cm}|p{3cm}|p{1.6cm}|p{2cm}|p{2cm}|}
 			\hline
 			Risk  & Effects on the Project  & Level Of Risk  & Plan B & Plan C  \\
 			\hline
 		\endfirsthead
 	
 			The technical algorithm planned for the project turned out to be an error in practice.  & Privacy, which is the purpose of the project, is at risk.  & High Risk  & Achieving practical solutions by checking software outputs. & Following the outputs through the UML diagram and detecting the error.  \\	\hline
 			Not noticing some "bugs" from the Testing Debugging phase.  & It may cause several problems in terms of user interface or software.  & Low-Moderate Risk  & Checking which line causes this situation. &   \\	\hline
 			Disruptions or signal confusion in the communication process between modules.  & Failure to communicate due to confusion may result in the process being completed incorrectly, or not being completed at all.  & Moderate-High Risk& By carrying out the process step by step, the source of the confusion can be found.  & In case of signal interference, distance can be left between the communication modules.  \\	\hline
 			The components we have do not work or cannot function at the time of the process. & It directly causes the project not to work.  & Low Risk  & The problem can be solved by making changes to the system using spare parts.  &   \\
 			\hline
 		
 		
 		
 	
 		\end{longtable}
 		\caption{Risk Analysis Table}
 	\end{table}
 Negative situations that may be encountered within the project have been identified. Although there are too many possibilities to list here, the basic technical problems that may be encountered are classified according to the risks listed below. The consequences of these problems for the project are shown in the table. Actions to be taken against these risks are included in the table as Plan B and Plan C. 
 	\newpage
 	\section{Cost Analysis}
 	\begin{table}[h!]
 		\centering
 		\begin{tabular}{|l|r|r|r|r|}
 			\hline
 			\textbf{Products} & \textbf{Quantity} & \textbf{Unit Price} & \textbf{Price} \\
 			\hline
 			STM32F411CEU6 & 2 & 224.00 TL & 448.00 TL \\ \hline
 			ESP8266       & 3 & 61.32 TL & 183.96 TL \\ \hline
 			USB - TTL     & 3 & 34.2 TL & 102.6 TL \\ \hline 
 			1S Li-Po Battery & 1 & 293.06 TL & 293.06 TL \\ \hline
 			Unexpected Expenses & 1 & 350 TL & 350 TL \\ 
 			\hline
 		\end{tabular}
 		\caption{Product Price List}
 	\end{table}
 	Our ElGamal Encryption System project, which we are preparing to launch, will play an important role in data security, which is one of the most important problems in today's world.
 	This developed project provides an advantage in terms of cost, which is an important problem in the marketing phase, with minimum budget and maximum engineering. The engineering principles underlying the project emphasize both user convenience and data security in the project.
 	This project, which has a very compact design, will appeal to all users in terms of its portability and ease of use.
 	This system, which has become much easier to control thanks to its designed user interface, has greatly expanded its audience and commercial range.
 	This system, which can appeal to customers of all ages and statuses, aims to easily find a place in the commercial world with a competitive budget based on this foundation.
 	In addition to being customizable according to customer demands, our design is also completely private and open to use on the basis of the defense industry. In this project, where software and engineering are actually marketed, our customer satisfaction target is kept very high.
 	Although this project, which is optimistic and prioritizes social welfare, will try to highlight itself with commercial advertisements, our main expectation is to obtain a customer portfolio based on customer satisfaction.
 	We hope that it can gain a commercial place thanks to these technical and usable advantages.
 	
 	\newpage
 		\section{Conclusions}
	
	Firstly, it is crucial to emphasize the purpose and requirements of the project. We plan to use two development boards to implement ElGamal encryption. One of these boards will perform the encryption process, while the other will decrypt the cipher. The user's input text will be encrypted for secure transmission. The primary objectives of our project include being cost-effective, appealing to a wide audience, and ensuring accessibility.\cite{koblitz2012course}

	
	Our proposed solution will employ the secure encryption algorithm ElGamal to ensure the security of communication. This differentiates our solution from other available alternatives because ElGamal is generally acknowledged as a reliable and effective encryption method\cite{mollin2001introduction}. Additionally, we offer a hardware-based solution using two different development boards, which can be considered an innovative approach to the project.
	
	ElGamal encryption stands out for its effectiveness and reliability, and the project's cost-effective solution is accessible to a wide range of users, according to the evaluations done. The hardware-based solution might provide more dependable and faster performance.
	
Cost-effectiveness is another important component that contributes to the project's attractiveness. Through the use of open-source technology and widely accessible development boards, the project manages expenses while maintaining a high standard of encryption service. This strategy fosters adoption and deployment in a variety of situations, from individual users concerned with privacy to small enterprises seeking for cost-effective security solutions, in addition to making the project financially accessible to a wide range of users.
	
In conclusion, there are a number of noteworthy benefits to our proposed solution, including its dependability, affordability, and appeal to a wide range of users. The use of the ElGamal encryption algorithm is seen as an effective method for ensuring communication security	\cite{van1999fundamentals}. Overall, the project shows promise as a general-purpose solution, and our hardware-based approach offers a unique benefit.
	\newpage
 \settocbibname{References}

\bibliographystyle{plain}
\bibliography{S1877050920308681} % Burada "referanslar" dosyasının adı

	\newpage
	\section{Appendix}
	\begin{figure}[H]
	\centering
	\label{Meeting4}
	\includegraphics[scale = 0.3]{images/meet4_02.01.2024.png}\\[0.5 cm]
				
\end{figure}
	\begin{figure}[H]
	\centering
	\label{Meeting5}
	\includegraphics[scale = 0.3]{images/meet4_10.01.2024.png}\\[0.5 cm]			
\end{figure}
	\begin{figure}[H]
	\centering
	\label{Meeting3}
	\includegraphics[scale = 0.3]{images/meet1.png}\\[0.5 cm]			
\end{figure}
	
	

\end{document}
