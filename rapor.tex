
\documentclass[12pt]{article}
\usepackage[english]{babel}
\usepackage{natbib}
\usepackage{url}
\usepackage[utf8x]{inputenc}
\usepackage{amsmath}
\usepackage{graphicx}
\graphicspath{{images/}}
\usepackage{parskip}
\usepackage{fancyhdr}
\usepackage{vmargin}
\usepackage[section]{placeins}
\usepackage{multirow}
\usepackage{multicol}
\usepackage{lipsum}
\usepackage{blindtext}
\usepackage{float}
\usepackage{nameref}
\setmarginsrb{3 cm}{0.5 cm}{3 cm}{2.5 cm}{1 cm}{1.5 cm}{1 cm}{1.5 cm}
\title{EEE 491 Engineering Design I}							
\date{\today}											

\makeatletter
\let\thetitle\@title
\let\theauthor\@author
\let\thedate\@date
\makeatother

\pagestyle{fancy}
\fancyhf{}
\rhead{\theauthor}
\lhead{\thetitle}
\cfoot{\thepage}

\begin{document}
	
	%%%%%%%%%%%%%%%%%%%%%%%%%%%%%%%%%%%%%%%%%%%%%%%%%%%%%%%%%%%%%%%%%%%%%%%%%%%%%%%%%%%%%%%%%
	
	\begin{titlepage}
		\centering
		\vspace*{0.5 cm}
		\includegraphics[scale = 0.75]{images/gazi.png}\\[1.0 cm]	
		\textsc{\LARGE Gazi University}\\[0.5 cm]
		\textsc{\LARGE Faculty of Engineering}\\[0.5 cm]
		\textsc{\LARGE Electrical Electronics Engineering}\\[1 cm]
		\textsc{\Large ElGamal Encryption System}\\[0.5 cm]			
		\rule{\linewidth}{0.5 mm} \\[0.4 cm]
		{ \huge \bfseries \thetitle}\\
		\rule{\linewidth}{0.5 mm} \\[1.5 cm]
		
		\begin{minipage}{0.4\textwidth}
			\begin{flushleft} \large
				\emph{Student Name(s):}\\
				\text{Bora Bostanoğlu}\\
				\text{Eren Öztürk}\\
				\text{Mert Afacan}\\
				
			\end{flushleft}
			
		\end{minipage}~
		\begin{minipage}{0.4\textwidth}
			\begin{flushright} \large
				\emph{Student Number:} \\				
				\text{191112012}\\
				\text{191112029}\\
				\text{191112001}\\
			% Your Student Number
			\end{flushright}
		\end{minipage}\\[2 cm]
			\begin{center}
			\textbf{Prof. Dr. Erkan Afacan}
		\end{center}
		{\large \thedate}\\[2 cm]
		
		\vfill
		
	\end{titlepage}
	%%%%%%%%%%%%%%%%%%%%%%%%%%%%%%%%%%%%%%%%%%%%%%%%%%%%%%%%%%%%%%%%%%%%%%%%%%%%%%%%%%%%%%%%%
	\tableofcontents
	\pagebreak
	
	%%%%%%%%%%%%%%%%%%%%%%%%%%%%%%%%%%%%%%%%%%%%%%%%%%%%%%%%%%%%%%%%%%%%%%%%%%%%%%%%%%%%%%%%%
	\pagenumbering{Roman} % İçindekilerin Roma rakamlarıyla başlaması için
	
	
	\renewcommand{\abstractname}{Summary}
	\begin{abstract}
		\addcontentsline{toc}{section}{Summary}		
		Bu bir özet metnidir. Buraya belgenizin ana hatları hakkında kısa bir özet yazabilirsiniz.
	\end{abstract}
	

\clearpage % Yeni bir sayfa başlatmak için
\pagenumbering{arabic}
	\section{Introduction}
Encryption is crucial in the digital world because it ensures that sensitive information remains secure and private. It involves encoding data in a way that only authorized parties can access and understand it. Without encryption, data such as personal information, financial details, passwords, and communications would be vulnerable to unauthorized access, theft, or manipulation by cybercriminals or malicious entities.

In today's landscape, encryption remains crucial as our reliance on digital communication and data sharing continues to grow. With the proliferation of smart devices, cloud storage, online transactions, and remote work, the volume of sensitive information transmitted over networks has increased significantly.

Moreover, as privacy concerns gain prominence, individuals seek assurance that their personal data, conversations, and transactions remain confidential. Encryption serves as a vital tool in protecting this information, especially in the face of evolving cybersecurity threats and sophisticated hacking techniques.

Overall, in today's digital era marked by increased connectivity and data sharing, encryption stands as an essential mechanism for safeguarding sensitive information, preserving privacy, and fortifying cybersecurity measures against evolving threats.

Additionally, in the realm of modern cryptography, various encryption methods are employed to ensure secure communication and data protection. These include well-established techniques like RSA and AES, along with innovative approaches such as Elliptic Curve Cryptography (ECC) for resource-constrained environments and Homomorphic Encryption for secure computation on encrypted data. ElGamal encryption, noted for its secure key exchange abilities, stands among these methods as a key player in ensuring the confidentiality and integrity of sensitive information in our increasingly digital world.

\vskip 20cm
	\section{Problem statement/Needs Identification}
	In today's interconnected digital world, secure communication stands as a fundamental necessity to safeguard sensitive information. However, ensuring data privacy during transmission across networks remains a significant challenge. The ElGamal encryption system, known for its robust security features based on the difficulty of discrete logarithm problem, presents an opportunity to establish secure communication channels.
		
	The primary objective involves designing an ElGamal-based system that enables the exchange of encrypted messages between two development board over an insecure communication channel. This system aims to ensure the confidentiality, integrity, and authenticity of the transmitted data. The project will delve into the intricacies of ElGamal's encryption and decryption processes within the context of development board communication.
	\section{Research/Technology Survey and/or Background information}
	
	\section{Requirements Specifications}
		
	\subsection{Engineering requirements}
	\begin{enumerate}
		\item[a.] Shall be communicate wireless.
		\item[b.] Shall encrypt/decrypt using at least 32 bit prime numbers.
		\item[c.] Shall has at least 1 hours operating time.
		\item[d.] Shall determine how the El-Gamal encryption algorithm will be implemented on the development board. This shall involve the mathematical operations of the algorithm, memory usage, and processor requirements. 
		\item[e.] Shall address security aspects such as the random number generator and key management necessary for El-Gamal encryption in a sensitive and secure manner.
		\item[f.] Shall create and implement comprehensive testing procedures to ensure the accuracy and reliability of encryption and decryption operations.
		\item[g.] Shall be able to operate consistently and effectively, without failing or
		experiencing downtime
				
	\end{enumerate}
	\subsection{Marketing Requirements}
	\begin{enumerate}
		\item[a.] Shall be cost-effective in its implementation.
		\item[b.] Shall have a user-friendly interface or provide adequate user support.
		\item[c.] Shall be usable in concealing data in every desired text.
		\item[d.] Shall be portable and moduler.
		\item[e.] Shall be dependable in ensuring user data security.				
	\end{enumerate}
	\subsection{Constraints}
		\subsubsection{Sustainability}
		\subsubsection{Environmental}
		\subsubsection{Economical}
		\subsubsection{Manufacturability}
		\subsubsection{Social}
  		Our project is subject to various social constraints such as respect for privacy, user consent, ethical use and security measures. The technologies used in the project must comply with legal regulations, and the transparent processing and protection of user data must be prioritized. User awareness should be raised through educational materials and community engagement.
	\subsection{Certifications and/or Standards}
	
	\section{Concept Generation and Evaluation}
	
	\subsection{Level 0 Design:}
	\begin{figure}[h]
		\centering
		\label{Level 0 Design of System }
		\includegraphics[scale = 0.15]{images/level0design.jpg}\\[0.5 cm]	
		\caption{Level 0 Design of System } 		
	\end{figure}
	\begin{table}[h]
		\centering
		
		\label{Properties Of Level 0 Design }
		\begin{tabular}{|c|c|}
			\hline
			Module & ElGamal Encryption/Decryption System \\ \hline
			\multirow{2}{*}{Inputs} & Power 3.3V DC \\ \cline{2-2}
			& User Input (Message) \\ \hline
			Output & Decrypted Message \\ \hline
			Functionality & Encrypting given input and  decrypting encrypted input \\ \hline
		
		\end{tabular}
		\caption{Properties of Level 0 Design }
	\end{table}
	
	\subsection{Level 1 Design}
	\begin{figure}[h]
		\centering
		\label{Level 1 Design of System }
		\includegraphics[scale = 0.25]{images/level1design.jpg}\\[0.5 cm]	
		\caption{Level 1 Design of System } 		
	\end{figure}
	\subsection{Demonstration of how behavioral models}
	\subsubsection{UML Design}
 	The general UML diagram of our software is shown in the figure and it shows that. First, a public key is generated on the decryption card for encryption and this key is sent to the encryption card. Then, on the encryption card, the user's message is converted into a numeric value and encrypted with the public key of the decryption card and the encrypted message is sent to the decryption card. The encrypted message is decrypted with the decryption card's own private key and the original message is obtained on the decryption card.
 	
 	\begin{figure}[h]
 		\centering
 		\label{Uml Diagram Of The System}
 		\includegraphics[scale = 0.25]{images/umldiagram.jpg}\\[0.5 cm]	
 		\caption{Uml Diagram Of The System} 		
 	\end{figure}
	\section{Conclusions}
	
	\section{References}
	
	\section{Appendix}


	
	

\end{document}
