
\documentclass[12pt]{article}
\usepackage[english]{babel}
\usepackage{natbib}
\usepackage{url}
\usepackage[utf8x]{inputenc}
\usepackage{amsmath}
\usepackage{graphicx}
\graphicspath{{images/}}
\usepackage{parskip}
\usepackage{fancyhdr}
\usepackage{vmargin}
\usepackage[section]{placeins}
\usepackage{multirow}
\usepackage{multicol}
\usepackage{lipsum}
\usepackage{blindtext}
\usepackage{float}
\usepackage{nameref}
\setmarginsrb{3 cm}{0.5 cm}{3 cm}{2.5 cm}{1 cm}{1.5 cm}{1 cm}{1.5 cm}
\title{EEE 491 Engineering Design I}							
\date{\today}											

\makeatletter
\let\thetitle\@title
\let\theauthor\@author
\let\thedate\@date
\makeatother

\pagestyle{fancy}
\fancyhf{}
\rhead{\theauthor}
\lhead{\thetitle}
\cfoot{\thepage}

\begin{document}
	
	%%%%%%%%%%%%%%%%%%%%%%%%%%%%%%%%%%%%%%%%%%%%%%%%%%%%%%%%%%%%%%%%%%%%%%%%%%%%%%%%%%%%%%%%%
	
	\begin{titlepage}
		\centering
		\vspace*{0.5 cm}
		\includegraphics[scale = 0.75]{images/gazi.png}\\[1.0 cm]	
		\textsc{\LARGE Gazi University}\\[0.5 cm]
		\textsc{\LARGE Faculty of Engineering}\\[0.5 cm]
		\textsc{\LARGE Electrical Electronics Engineering}\\[1 cm]
		\textsc{\Large ElGamal Encryption System}\\[0.5 cm]			
		\rule{\linewidth}{0.5 mm} \\[0.4 cm]
		{ \huge \bfseries \thetitle}\\
		\rule{\linewidth}{0.5 mm} \\[1.5 cm]
		
		\begin{minipage}{0.4\textwidth}
			\begin{flushleft} \large
				\emph{Student Name(s):}\\
				\text{Bora Bostanoğlu}\\
				\text{Eren Öztürk}\\
				\text{Mert Afacan}\\
				
			\end{flushleft}
			
		\end{minipage}~
		\begin{minipage}{0.4\textwidth}
			\begin{flushright} \large
				\emph{Student Number:} \\				
				\text{191112012}\\
				\text{191112029}\\
				\text{191112001}\\
			% Your Student Number
			\end{flushright}
		\end{minipage}\\[2 cm]
			\begin{center}
			\textbf{Prof. Dr. Erkan Afacan}
		\end{center}
		{\large \thedate}\\[2 cm]
		
		\vfill
		
	\end{titlepage}
	%%%%%%%%%%%%%%%%%%%%%%%%%%%%%%%%%%%%%%%%%%%%%%%%%%%%%%%%%%%%%%%%%%%%%%%%%%%%%%%%%%%%%%%%%
	\tableofcontents
	\pagebreak
	
	%%%%%%%%%%%%%%%%%%%%%%%%%%%%%%%%%%%%%%%%%%%%%%%%%%%%%%%%%%%%%%%%%%%%%%%%%%%%%%%%%%%%%%%%%
	\pagenumbering{Roman} % İçindekilerin Roma rakamlarıyla başlaması için
	
	
	\renewcommand{\abstractname}{Summary}
	\begin{abstract}
		\addcontentsline{toc}{section}{Summary}		
		Data security is very important in today's world. As a result of many malicious attempts, the need to ensure the security of digital data has arisen. In this context, there are many encryption methods to protect and secure digital data. In this project, ElGamal encryption method was used for the security of data exchange.
		
		The aim of our project is to ensure software security and prevent malicious attempts by using the ElGamal encryption method during data transfer.
		
		The software we want to create within the scope of this project; The message sent by the sender must be transmitted by the wi-fi module to our encryption development card, and then the sent message must be encrypted by the encryption development card using the ElGamal encryption method. This encrypted message must then be transmitted to the wi-fi module and directed to our decryption development card so that the text can be delivered deciphered to the recipient. The encrypted message deciphered on this card is transmitted back to the wi-fi module. Then, the deciphered message is transmitted to the recipient via this module, ensuring secure communication.
		
		In this way, data security will be ensured and one of the problems that has a large share in today's digitalizing world will be prevented.
		
	\end{abstract}
	

\clearpage % Yeni bir sayfa başlatmak için
\pagenumbering{arabic}
	\section{Introduction}
Encryption is crucial in the digital world because it ensures that sensitive information remains secure and private. It involves encoding data in a way that only authorized parties can access and understand it. Without encryption, data such as personal information, financial details, passwords, and communications would be vulnerable to unauthorized access, theft, or manipulation by cybercriminals or malicious entities.

In today's landscape, encryption remains crucial as our reliance on digital communication and data sharing continues to grow. With the proliferation of smart devices, cloud storage, online transactions, and remote work, the volume of sensitive information transmitted over networks has increased significantly.

Moreover, as privacy concerns gain prominence, individuals seek assurance that their personal data, conversations, and transactions remain confidential. Encryption serves as a vital tool in protecting this information, especially in the face of evolving cybersecurity threats and sophisticated hacking techniques.

Overall, in today's digital era marked by increased connectivity and data sharing, encryption stands as an essential mechanism for safeguarding sensitive information, preserving privacy, and fortifying cybersecurity measures against evolving threats.

Additionally, in the realm of modern cryptography, various encryption methods are employed to ensure secure communication and data protection. These include well-established techniques like RSA and AES, along with innovative approaches such as Elliptic Curve Cryptography (ECC) for resource-constrained environments and Homomorphic Encryption for secure computation on encrypted data. ElGamal encryption, noted for its secure key exchange abilities, stands among these methods as a key player in ensuring the confidentiality and integrity of sensitive information in our increasingly digital world.

\vskip 20cm
	\section{Needs Identification}
It is essential to identify the requirements concerning two development boards intended for El Gamal encryption. These boards should be designed as a pair, one dedicated to encryption and the other for decrypting the encrypted data.


The design of the two development boards intended for El Gamal encryption must facilitate encryption and decryption processes to ensure data security in communication. These boards should possess reliable and effective capabilities for data encryption and decryption. Moreover, they should encompass algorithms and processes necessary to ensure data integrity and security.


This project entails the design and development of two dedicated boards. The primary goal is to create systems capable of effectively performing El Gamal encryption, enabling one board to perform encryption and the other to decipher the encrypted data, thereby ensuring secure communication. These boards should operate in compliance with industry standards to ensure data security and confidentiality.


	\section{Research Survey and Background information}
		
		ElGamal Encryption: ElGamal is a public-key(asymmetric) cryptosystem based on the mathematical properties of modular exponentiation and the difficulty of certain mathematical problems. It involves key generation, encryption, and decryption processes.
		How It's Currently Being Done:
		
		Key Generation: Each card generates its own public and private key pair.
		Encryption: Card A encrypts a message using Card B's public key and random integer number.
		
		Decryption: Card B decrypts the received message using its private key.
		Limitations of Current Designs or Technology:
		
		Computational Intensity: ElGamal encryption can be computationally intensive, which might be a limitation for resource-constrained devices.
		
		Key Size: To ensure security, ElGamal often requires larger key sizes, potentially impacting communication efficiency.
		
		Vulnerability to Quantum Attacks: ElGamal, like many public-key systems, is vulnerable to attacks leveraging quantum computers.
		
		
		In an article published by Omar A. Imran in 2019 they focused on speech signal encryption/decryption.
		
		 The referenced article primarily focuses on employing El-Gamal encryption for speech signal encryption and decryption. In contrast, our project concentrates on the development of dual-board systems specifically for El-Gamal encryption, with specific data type which is text.
	 
		The differences of our project from the mentioned model are as follows:
		
		Dual-Board Encryption System: Unlike some existing projects that might focus on single-board encryption, our project distinctly involves the development of a dual-board system specifically designed for El Gamal encryption. This two-board setup, with one dedicated to encryption and the other for decryption, is a unique aspect of our project.
		
		
		The similarities of our project with the mentioned article are as follows:
		
		Focus on Data Security: Similar to several existing projects, our project shares a fundamental focus on ensuring data security. Encryption and decryption processes aim to safeguard transmitted data from unauthorized access.
			
		Utilization of El-Gamal Algorithm: Both the referenced article and our project utilize the El-Gamal algorithm for encryption purposes, acknowledging its strength in public key cryptography.
		

		
	
	
	\section{Requirements Specifications}
		
	\subsection{Engineering requirements}
	\begin{enumerate}
		\item[a.] Shall be communicate wireless.
		\item[b.] Shall encrypt/decrypt using at least 32 bit prime numbers.
		\item[c.] Shall has at least 1 hours operating time.
		\item[d.] Shall determine how the El-Gamal encryption algorithm will be implemented on the development board. This shall involve the mathematical operations of the algorithm, memory usage, and processor requirements. 
		\item[e.] Shall address security aspects such as the random number generator and key management necessary for El-Gamal encryption in a sensitive and secure manner.
		\item[f.] Shall create and implement comprehensive testing procedures to ensure the accuracy and reliability of encryption and decryption operations.
		\item[g.] Shall be able to operate consistently and effectively, without failing or
		experiencing downtime
				
	\end{enumerate}
	\subsection{Marketing Requirements}
	\begin{enumerate}
		\item[a.] Shall be cost-effective in its implementation.
		\item[b.] Shall have a user-friendly interface or provide adequate user support.
		\item[c.] Shall be usable in concealing data in every desired text.
		\item[d.] Shall be portable and moduler.
		\item[e.] Shall be dependable in ensuring user data security.				
	\end{enumerate}
	\subsection{Constraints}
		\subsubsection{Sustainability}
		ElGamal Encryption may involve complex mathematical operations, and the algorithms used for encryption and decryption can consume significant computational resources. It's important to consider the sustainability of the system in terms of energy efficiency, especially if it is deployed in resource-constrained environments.
		\subsubsection{Environmental}
		Our environmental goal in this encryption project is to mitigate collateral damage associated with vulnerability that can affect environment. While encryption is essential for data security, it can inadvertently lead to collateral risks such as vulnerabilities or access limitations. Our focus is mitigate any potential adverse effects that might arise as collateral damage.
		\subsubsection{Economical}
		The ElGamal encryption implementation requires the use of cost-effective components crucial for encryption and communication purposes. This economic analysis aims to outline the cost estimates associated with these components, emphasizing the utilization of budget-friendly options.
			
		The ElGamal encryption project's economic aspect estimates a total cost of 20\$ for the development boards. Emphasis on utilizing budget-friendly components is highlighted for an economical solution. Further detailing of any additional expenses is recommended for a comprehensive economic evaluation.
		\subsubsection{Manufacturability}
	Constraints related to manufacturability should address the ease of integrating ElGamal Encryption into existing systems and technologies. Ensuring compatibility and interoperability with different platforms and software solutions is crucial for seamless adoption and widespread use.
		\subsubsection{Social}
  		Our project is subject to various social constraints such as respect for privacy, user consent, ethical use and security measures. The technologies used in the project must comply with legal regulations, and the transparent processing and protection of user data must be prioritized. User awareness should be raised through educational materials and community engagement.
	\subsection{Certifications and/or Standards}
	In our pursuit of developing a robust ElGamal encryption solution, we are dedicated to integrating industry-established standards into our hardware and embedded software development practices. The incorporation of IEEE 1413 for hardware description languages and IEEE/ISO/IEC 14764 for software engineering processes serves as the cornerstone of our commitment. IEEE 1413 guides our hardware design methodologies, ensuring consistency and efficiency in utilizing hardware description languages for electronic systems. Simultaneously, IEEE/ISO/IEC 14764 shapes our software engineering processes, emphasizing meticulous software maintenance, management, and configuration control practices throughout the lifecycle. By adhering to these standards, we aim to fortify the reliability, compatibility, and sustainability of our ElGamal encryption project, assuring a high-quality, industry-aligned solution.
	\section{Concept Generation and Evaluation}
	
	\subsection{Level 0 Design:}
	\begin{figure}[H]
		\centering
		\label{Level 0 Design of System }
		\includegraphics[scale = 0.15]{images/level0design.jpg}\\[0.5 cm]	
		\caption{Level 0 Design of System } 		
	\end{figure}
	\begin{table}[H]
		\centering
		
		\label{Properties Of Level 0 Design }
		\begin{tabular}{|c|c|}
			\hline
			Module & ElGamal Encryption/Decryption System \\ \hline
			\multirow{2}{*}{Inputs} & Power 3.3V DC \\ \cline{2-2}
			& User Input (Message) \\ \hline
			Output & Decrypted Message \\ \hline
			Functionality & Encrypting given input and  decrypting encrypted input \\ \hline
		
		\end{tabular}
		\caption{Properties of Level 0 Design }
	\end{table}
	
	\subsection{Level 1 Design}
	
	\begin{figure}[H]
		\centering
		\label{Level 1 Design of System }
		\includegraphics[scale = 0.25]{images/level1design.jpg}\\[0.5 cm]	
		\caption{Level 1 Design of System } 		
	\end{figure}
\begin{table}[h]
	\centering
	
	\label{Properties of Encryption Card}
	\begin{tabular}{|c|c|}
		\hline
		Module & Encryption Card \\ \hline
		\multirow{3}{*}{Inputs} & Power 3.3V DC \\
		\cline{2-2}
		& User Input (Message) \\
		\cline{2-2}
		& Public Key \\ \hline
		Output & Encrypted Message \\ \hline
		Functionality & Encrypting given input.\\ \hline		
	\end{tabular}
	\caption{Properties of Encryption Card}
\end{table}

\begin{table}[h]
	\centering	
	\label{Properties of Decryption Card }
	\begin{tabular}{|c|c|}
	\hline
	Module & Decryption Card\\ \hline
		\multirow{2}{*}{Inputs} & Power 3.3V DC \\ \cline{2-2}
	& Encrypted Message \\ \hline
		\multirow{2}{*}{Outputs} & Public Key \\ \cline{2-2}
	& Decrypted Message \\ \hline
	Functionality & Decrypting given input.\\ \hline	
		
	\end{tabular}
	\caption{Properties of Decryption Card}
\end{table}
	\begin{table}[h]
		\centering		
		\label{Properties of Access Point }
		\begin{tabular}{|c|c|}
			\hline
			Module & Access Point \\ \hline
			\multirow{5}{*}{Inputs} & Power 3.3V DC \\
			\cline{2-2}
			& User Input (Message) \\
			\cline{2-2}
				& Encrypted Message \\
			\cline{2-2}
				& Decrypted Message \\
			\cline{2-2}
				& Public Key \\
			\cline{2-2} \hline
			
				\multirow{4}{*}{Outputs} & Message\\
			\cline{2-2}
			& Public Key \\
			\cline{2-2}
			& Encrypted Message \\
			\cline{2-2}
			& Decrypted Message \\
			\cline{2-2} \hline
			
			Functionality & Access Point.\\ \hline	
			
		\end{tabular}
		\caption{Properties of Access Point}
	\end{table}



\vskip 10cm
	\subsection{Demonstration of how behavioral models}
	\subsubsection{UML Design}
 	The general UML diagram of our software is shown in the figure and it shows that. First, a public key is generated on the decryption card for encryption and this key is sent to the encryption card. Then, on the encryption card, the user's message is converted into a numeric value and encrypted with the public key of the decryption card and the encrypted message is sent to the decryption card. The encrypted message is decrypted with the decryption card's own private key and the original message is obtained on the decryption card.
 	
 	\begin{figure}[h]
 		\centering
 		\label{Uml Diagram Of The System}
 		\includegraphics[scale = 0.25]{images/umldiagram.jpg}\\[0.5 cm]	
 		\caption{Uml Diagram Of The System} 		
 	\end{figure}
	\section{Conclusions}
	
	Our project is aimed at data security, which is a deep-rooted problem. First of all, we investigated the reason for the emergence of a branch of science such as Cryptography and how this need was integrated into the digital world as a result of historical developments.
	
	As a result of the research, we have seen that there are many encryption and security methods.
	Later, we decided to continue with the ElGamal encryption method, which we thought was more suitable for our system.
	
	Within the scope of this project, we first learned about ElGamal and Cryptography with the support of our consultant. Then, we started to determine the requirements for the system we would implement. At this stage, we created Level 0 and Level 1 designs for the preliminary draft report. First of all, we performed the basic modeling of the system and determined the communication and system message transfer points. Afterwards, we defined the Wi-fi module and development cards that should be used for this system in more detail. Within this range of components, we defined which direction the data connection should be at which stage.
	
	Within the UML design, we have defined in more detail the sequences in which the system will perform these encryption stages. We have completed the basic lines of our project in the draft design of this system, in which we are technically more advanced.
	
	In the later stages, we will move towards technically detailed modeling of the system and definition of the software, such as Level 2. We will determine exactly the development cards we will use and try to act according to the points we mentioned regarding restrictions.
	
	\section{References}
	
	\section{Appendix}
	\begin{figure}[H]
	\centering
	\label{Meeting1}
	\includegraphics[scale = 0.5]{images/meet3.png}\\[0.5 cm]			
\end{figure}
	\begin{figure}[H]
	\centering
	\label{Meeting2}
	\includegraphics[scale = 0.5]{images/meet2.png}\\[0.5 cm]			
\end{figure}
	\begin{figure}[H]
	\centering
	\label{Meeting3}
	\includegraphics[scale = 0.5]{images/meet1.png}\\[0.5 cm]			
\end{figure}
	
	

\end{document}
