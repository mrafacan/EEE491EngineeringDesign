
\documentclass[12pt]{article}
\usepackage[english]{babel}
\usepackage{natbib}
\usepackage{url}
\usepackage[utf8x]{inputenc}
\usepackage{enumitem}
\usepackage{amsmath}
\usepackage{graphicx}
\graphicspath{{images/}}
\usepackage{parskip}
\usepackage{fancyhdr}
\usepackage{vmargin}
\usepackage[section]{placeins}
\usepackage{multirow}
\usepackage{multicol}
\usepackage{lipsum}
\usepackage{blindtext}
\usepackage{amssymb}
\usepackage[nottoc,numbib]{tocbibind}
\usepackage{float}
\usepackage{nameref}
\setmarginsrb{3 cm}{0.5 cm}{3 cm}{2.5 cm}{1 cm}{1.5 cm}{1 cm}{1.5 cm}
\title{EEE 491 Engineering Design I}							
\date{\today}											

\makeatletter
\let\thetitle\@title
\let\theauthor\@author
\let\thedate\@date
\makeatother

\pagestyle{fancy}
\fancyhf{}
\rhead{\theauthor}
\lhead{\thetitle}
\cfoot{\thepage}

\begin{document}
	
	%%%%%%%%%%%%%%%%%%%%%%%%%%%%%%%%%%%%%%%%%%%%%%%%%%%%%%%%%%%%%%%%%%%%%%%%%%%%%%%%%%%%%%%%%
	
	\begin{titlepage}
		\centering
		\vspace*{0.5 cm}
		\includegraphics[scale = 2.85]{images/gazi.png}\\[1.0 cm]	
		\textsc{\LARGE Gazi University}\\[0.5 cm]
		\textsc{\LARGE Faculty of Engineering}\\[0.5 cm]
		\textsc{\LARGE Electrical Electronics Engineering}\\[1 cm]
		\textsc{\Large ElGamal Encryption System}\\[0.5 cm]			
		\rule{\linewidth}{0.5 mm} \\[0.4 cm]
		{ \huge \bfseries \thetitle}\\
		\rule{\linewidth}{0.5 mm} \\[1.5 cm]
		
		\begin{minipage}{0.4\textwidth}
			\begin{flushleft} \large
				\emph{Student Name(s):}\\
				\text{Bora Bostanoğlu}\\
				\text{Eren Öztürk}\\
				\text{Mert Afacan}\\
				
			\end{flushleft}
			
		\end{minipage}~
		\begin{minipage}{0.4\textwidth}
			\begin{flushright} \large
				\emph{Student Number:} \\				
				\text{191112012}\\
				\text{191112029}\\
				\text{191112001}\\
			% Your Student Number
			\end{flushright}
		\end{minipage}\\[2 cm]
			\begin{center}
			\textbf{Prof. Dr. Erkan Afacan}
		\end{center}
		{\large \thedate}\\[2 cm]
		
		\vfill
		
	\end{titlepage}
	%%%%%%%%%%%%%%%%%%%%%%%%%%%%%%%%%%%%%%%%%%%%%%%%%%%%%%%%%%%%%%%%%%%%%%%%%%%%%%%%%%%%%%%%%
	\tableofcontents
	\thispagestyle{empty}
	\pagebreak
	
	%%%%%%%%%%%%%%%%%%%%%%%%%%%%%%%%%%%%%%%%%%%%%%%%%%%%%%%%%%%%%%%%%%%%%%%%%%%%%%%%%%%%%%%%%
	\pagenumbering{Roman} % İçindekilerin Roma rakamlarıyla başlaması için
	
	
	\renewcommand{\abstractname}{Summary}
	\begin{abstract}
		\addcontentsline{toc}{section}{Summary}				
		With the goal of bridging the gap between robust security and affordability, this project introduces an innovative system that uses ElGamal encryption to make available confidential messaging in response to the growing demand for accessible secure communication. This system allows individuals from diverse backgrounds to participate in secure text-based interactions. After developing the encryption algorithm and identifying particular hardware components—such as STM32 microcontrollers, a Wi-Fi module for data transmission, and USB interfaces for connectivity—the project has reached important milestones. The system runs on two STM32 development boards, one for encryption and the other for decryption, providing an efficient and streamlined method.The ultimate goal is to redefine secure messaging, guaranteeing that people can interact with confidence while protecting the confidentiality of their conversations. This initiative looks to make secure communication a daily reality by utilizing these widely compatible and easily accessible hardware components.
		
	\end{abstract}
	

\clearpage % Yeni bir sayfa başlatmak için
\pagenumbering{arabic}
	\section{Introduction}
In our era, ensuring the confidentiality and security of our communication is important. However, achieving this security often comes with a unaffordable price tag, makes it inaccessible to many. Recognizing the need for a cost-effective yet robust solution, we introduce a system designed to make secure messaging accessible to everyone using the efficient ElGamal encryption method.

This project's main objective is to create a safe communication environment that is open to all. We do this by utilizing ElGamal encryption, which is widely known for its efficiency and low cost. We hope to remove obstacles and offer a platform for private text messages without putting a heavy financial strain on users.

Two development boards function as the system's basic yet efficient mechanism: one board encrypts the data, guaranteeing its privacy, and the other board decrypts it, enabling safe delivery of the initial message. This method guarantees both affordability and ease of use, enabling secure communication to be a reality for people from a variety of backgrounds.

By utilizing these affordable development boards, we hope to increase the security of messages sent and received while maintaining the integrity of the encryption. In this way, we hope to integrate secure communication into daily interactions and promote a more inclusive and safe online community.

At its core, our project seeks to reimagine secure communication by filling the gap that exists between effective encryption and affordability. We foresee a future in which people can engage with confidence, knowing that their messages are private and secure, because secure messaging will become more widely accessible.


\vskip 20cm
	\section{Needs Identification}

For text-based encryption, there is a need to create a system that enables secure messaging for everyone. Utilizing the cost-effective ElGamal encryption method, this system aims to provide accessible and secure communication for individuals seeking confidential text-based interaction without incurring significant expenses. The system operates through two development boards, where one board encrypts the data and sends it to the other board, responsible for decryption. This cost-efficient approach ensures that anyone can securely exchange messages, fostering a safer digital environment for communication.

	
	\subsection{Statement of Engineering Problem:}
	The engineering challenge lies in developing a system that efficiently enables text-based encryption for secure messaging without imposing high costs. The existing solutions might either lack accessibility or be cost-prohibitive for many users. The problem at hand involves creating an encryption solution that ensures confidentiality in text-based communication without being financially burdensome.
	
	\subsection{Goal and Objectives:}
	\subsubsection{Goal:}
	The primary objective is to create a cost-efficient system utilizing the ElGamal encryption method, enabling accessible and secure text-based communication for all users.
	
	\subsubsection{Objectives:}
	\begin{itemize}
		\item Develop a system that implements ElGamal encryption for text-based communication securely.
		\item Ensure the system is cost-effective, avoiding significant financial barriers for users.
		\item Design the system to operate between two development boards, enabling encryption and decryption seamlessly.
		\item Provide a means for individuals to securely exchange messages without compromising confidentiality.
		\item Foster a safer digital environment by making secure communication accessible to everyone.
	\end{itemize}
\newpage
	\section{Research Survey and Background information}

	\begin{itemize}
		\item \textbf{ElGamal Encryption}:
		\begin{itemize}
			\item ElGamal is a public-key (asymmetric) cryptosystem based on the mathematical properties of modular exponentiation and the difficulty of certain mathematical problems\cite{book}.
			\item It involves key generation, encryption, and decryption processes.
		\end{itemize}
		
		\item \textbf{How It's Currently Being Done}:
		\begin{itemize}
			\item \textbf{Key Generation}: Each card generates its own public and private key pair.
			\item \textbf{Encryption}: Card A encrypts a message using Card B's public key and a random integer number.
			\item \textbf{Decryption}: Card B decrypts the received message using its private key.
		\end{itemize}
		
		\item \textbf{Limitations of Current Designs or Technology}:
		\begin{itemize}
			\item \textbf{Computational Intensity}: ElGamal encryption can be computationally intensive, which might be a limitation for resource-constrained devices.
			\item \textbf{Key Size}: To ensure security, ElGamal often requires larger key sizes, potentially impacting communication efficiency.
			\item \textbf{Vulnerability to Quantum Attacks}: ElGamal, like many public-key systems, is vulnerable to attacks leveraging quantum computers.
		\end{itemize}
	\end{itemize}
		
		
		In an article published by Omar A. Imran in 2019\cite{IMRAN20201028} they focused on speech signal encryption/decryption.	The referenced article primarily focuses on employing El-Gamal encryption for speech signal encryption and decryption. 
		 
	 \textbf{The differences of our project from the mentioned model are as follows:}
	 \begin{itemize}
	 	\item \textbf{Dual-Board Encryption System}: Unlike some existing projects that might focus on single-board encryption, our project distinctly involves the development of a dual-board system specifically designed for ElGamal encryption.
	 \end{itemize}
	 
	 \textbf{The similarities of our project with the mentioned article are as follows:}
	 \begin{itemize}
	 	\item \textbf{Focus on Data Security}: Similar to several existing projects, our project shares a fundamental focus on ensuring data security. Encryption and decryption processes aim to safeguard transmitted data from unauthorized access.
	 	
	 	\item \textbf{Utilization of ElGamal Algorithm}: Both the referenced article and our project utilize the ElGamal algorithm for encryption purposes, acknowledging its strength in public key cryptography.
	 \end{itemize}
		

		
	
	
	\section{Requirements Specifications}
		
	\subsection{Engineering requirements}
	\begin{enumerate}
		\item[a.] Shall communicate eachother wireless.
		\item[b.] Shall communicate users with USB.
		\item[c.] Shall encrypt/decrypt using at least 32 bit prime numbers.
		\item[d.] Shall has at least 1 hours operating time.
		\item[e.] Shall determine how the El-Gamal encryption algorithm will be implemented on the development board. This shall involve the mathematical operations of the algorithm, memory usage, and processor requirements. 
		\item[f.] Shall address security aspects such as the random number generator and key management necessary for El-Gamal encryption in a sensitive and secure manner.
		
	
				
	\end{enumerate}
	\subsection{Marketing Requirements}
	\begin{enumerate}
		\item[a.] Shall  cost-effective in its implementation.
		\item[b.] Shall have a user-friendly interface or provide adequate user support.
		\item[c.] Shall  usable in concealing data in every desired text.
		\item[d.] Shall  portable and moduler.
		\item[e.] Shall  dependable in ensuring user data security.				
	\end{enumerate}
	\subsection{Constraints}
		\subsubsection{Sustainability}
		ElGamal Encryption may involve complex mathematical operations, and the algorithms used for encryption and decryption can consume significant computational resources. It's important to consider the sustainability of the system in terms of energy efficiency, especially if it is deployed in resource-constrained environments.
		\subsubsection{Environmental}
		Our environmental goal in this encryption project is to mitigate collateral damage associated with vulnerability that can affect environment. While encryption is essential for data security, it can inadvertently lead to collateral risks such as vulnerabilities or access limitations. Our focus is mitigate any potential adverse effects that might arise as collateral damage.
		\subsubsection{Economical}
		The ElGamal encryption implementation requires the use of cost-effective components crucial for encryption and communication purposes. This economic analysis aims to outline the cost estimates associated with these components, emphasizing the utilization of budget-friendly options.
			
		The ElGamal encryption project's economic aspect estimates a total cost of 20\$ for the development boards. Emphasis on utilizing budget-friendly components is highlighted for an economical solution. Further detailing of any additional expenses is recommended for a comprehensive economic evaluation.
		\subsubsection{Manufacturability}
	Constraints related to manufacturability should address the ease of integrating ElGamal Encryption into existing systems and technologies. Ensuring compatibility and interoperability with different platforms and software solutions is crucial for seamless adoption and widespread use.
		\subsubsection{Social}
  		Our project is subject to various social constraints such as respect for privacy, user consent, ethical use and security measures. The technologies used in the project must comply with legal regulations, and the transparent processing and protection of user data must be prioritized. User awareness should be raised through educational materials and community engagement.
	\subsection{Certifications and/or Standards}
	In our pursuit of developing a robust ElGamal encryption solution, we are dedicated to integrating industry-established standards into our hardware and embedded software development practices. The incorporation of IEEE 1413 for hardware description languages, ISO/IEC 27001 international standard to manage information security and IEEE/ISO/IEC 14764 for software engineering processes serves as the cornerstone of our commitment. IEEE 1413 guides our hardware design methodologies, ensuring consistency and efficiency in utilizing hardware description languages for electronic systems. Simultaneously, IEEE/ISO/IEC 14764 shapes our software engineering processes, emphasizing meticulous software maintenance, management and configuration control practices throughout the lifecycle. By adhering to these standards, we aim to fortify the reliability, compatibility and sustainability of our ElGamal encryption project, assuring a high-quality, industry-aligned solution.
	\section{Concept Generation and Evaluation}
	
	\subsection{Level 0 Design:}
	\begin{figure}[H]
		\centering
		\label{Level 0 Design of System }
		\includegraphics[scale = 0.15]{images/level0design.jpg}\\[0.5 cm]	
		\caption{Level 0 Design of System } 		
	\end{figure}
	\begin{table}[H]
		\centering
		
		\label{Properties Of Level 0 Design }
		\begin{tabular}{|c|c|}
			\hline
			Module & ElGamal Encryption/Decryption System \\ \hline
			\multirow{2}{*}{Inputs} & Power 3.3V DC \\ \cline{2-2}
			& User Input (Message) \\ \hline
			Output & Decrypted Message \\ \hline
			Functionality & Encrypting given input and  decrypting encrypted input \\ \hline
		
		\end{tabular}
		\caption{Properties of Level 0 Design }
	\end{table}
	
	\subsection{Level 1 Design:}

	\begin{figure}[H]
		\centering
		\label{Level 1 Design of System }
		\includegraphics[scale = 0.25]{images/level1design.jpg}\\[0.5 cm]	
		\caption{Level 1 Design of System } 		
	\end{figure}
	

\begin{table}[h]

	\centering
	
	\label{Properties of Encryption Card}
	\begin{tabular}{|c|c|}
		\hline
		Module & Encryption Card \\ \hline
		\multirow{3}{*}{Inputs} & Power 3.3V DC \\
		\cline{2-2}
		& User Input (Message) \\
		\cline{2-2}
		& Public Key \\ \hline
		Output & Encrypted Message \\ \hline
		Functionality & Encrypting given input.\\ \hline		
	\end{tabular}
	\caption{Properties of Encryption Card}

\end{table}

\begin{table}[h]

	\centering	
	\label{Properties of Decryption Card }
	\begin{tabular}{|c|c|}
	\hline
	Module & Decryption Card\\ \hline
		\multirow{2}{*}{Inputs} & Power 3.3V DC \\ \cline{2-2}
	& Encrypted Message \\ \hline
		\multirow{2}{*}{Outputs} & Public Key \\ \cline{2-2}
	& Decrypted Message \\ \hline
	Functionality & Decrypting given input.\\ \hline	
		
	\end{tabular}
	\caption{Properties of Decryption Card}

\end{table}
	\begin{table}[H]

		\centering		
		\label{Properties of Access Point }
		\begin{tabular}{|c|c|}
			\hline
			Module & Access Point \\ \hline
			\multirow{3}{*}{Inputs} & Power 3.3V DC \\
			
			\cline{2-2}
				& Encrypted Message \\
		
			\cline{2-2}
				& Public Key \\
			\cline{2-2} \hline
	
				\multirow{2}{*}{Outputs} 	& Public Key \\
			\cline{2-2}
			& Encrypted Message \\
			
			\cline{2-2} \hline
			
			Functionality & Access Point.\\ \hline	
	
		\end{tabular}
		\caption{Properties of Access Point}

	\end{table}
	
	\begin{table}[H]
	\centering
	\begin{tabular}{|c|c|c|c|c|}
		\hline
		UI & Display & Connection & Power & Dev Board \\
		\hline
		\textbf{UI App}$\checkmark$
		 &\textbf{Monitor}$\checkmark$
		   & \textbf{Wifi}$\checkmark$
		   & Solar Cell & Arduino\\
		\hline
		\textbf{Keyboard}$\checkmark$
		 & LCD & Bluetooth & Li-on Battery & Rasberry Pi \\
		\hline
		Voice & Head-up Display & Ethernet & Fuel Cell &\textbf{Stm32}$\checkmark$
		 \\
		\hline
		Video & Oled &\textbf{USB-TTL}$\checkmark$  & \textbf{Lipo Battery}$\checkmark$
		 & Jetson Nano \\
		\hline
	\end{tabular}
	\caption{Consept Table}
\end{table}


\vskip 10cm
\subsection{Demonstration of how behavioral models:}
	
	\subsubsection{Activity Diagram}
 The general activity diagram of our project is shown in the Figure 3 and it shows that. First, a public key is generated on the decryption card for encryption and this key is sent to the encryption card. Meanwhile user can enter a message to the user interface and send it to the encryption card. Then, on the encryption card, the user's message is converted into a numeric value and encrypted with the public key of the receiver card and the encrypted message is sent to the decryption card. on the decryption card, the encrypted message is decrypted with the decryption card's own private key and the original message is obtained. Finally decryption card sends message to the user interface and user can display the message.
 	\begin{figure}[H]
 		\centering
 		\label{Uml Diagram Of The System}
 		\includegraphics[scale = 0.35]{images/activitydiagram.jpg}\\[0.5 cm]	
 		\caption{Activity Diagram Of The System} 		
 	\end{figure}
 	\newpage
	\section{Conclusions}
	
	Firstly, it is crucial to emphasize the purpose and requirements of the project. We plan to use two development boards to implement ElGamal encryption. One of these boards will perform the encryption process, while the other will decrypt the cipher. The user's input text will be encrypted for secure transmission. The primary objectives of our project include being cost-effective, appealing to a wide audience, and ensuring accessibility.\cite{koblitz2012course}

	
	Our proposed solution will employ the secure encryption algorithm ElGamal to ensure the security of communication. This differentiates our solution from other available alternatives because ElGamal is generally acknowledged as a reliable and effective encryption method\cite{mollin2001introduction}. Additionally, we offer a hardware-based solution using two different development boards, which can be considered an innovative approach to the project.
	
	ElGamal encryption stands out for its effectiveness and reliability, and the project's cost-effective solution is accessible to a wide range of users, according to the evaluations done. The hardware-based solution might provide more dependable and faster performance.
	
In conclusion, there are a number of noteworthy benefits to our proposed solution, including its dependability, affordability, and appeal to a wide range of users. The use of the ElGamal encryption algorithm is seen as an effective method for ensuring communication security	\cite{van1999fundamentals}. Overall, the project shows promise as a general-purpose solution, and our hardware-based approach offers a unique benefit.
	\newpage
 \settocbibname{References}

\bibliographystyle{plain}
\bibliography{S1877050920308681} % Burada "referanslar" dosyasının adı

	\newpage
	\section{Appendix}
	\begin{figure}[H]
	\centering
	\label{Meeting1}
	\includegraphics[scale = 0.3]{images/meet3.png}\\[0.5 cm]			
\end{figure}
	\begin{figure}[H]
	\centering
	\label{Meeting2}
	\includegraphics[scale = 0.3]{images/meet2.png}\\[0.5 cm]			
\end{figure}
	\begin{figure}[H]
	\centering
	\label{Meeting3}
	\includegraphics[scale = 0.3]{images/meet1.png}\\[0.5 cm]			
\end{figure}
	
	

\end{document}
